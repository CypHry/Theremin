% !TeX encoding = UTF-8
% !TeX spellcheck = pl_PL

% $Id:$

%Author: Wojciech Domski
%Szablon do ząłożeń projektowych, raportu i dokumentacji z steorwników robotów
%Wersja v.1.0.0
%


%% Konfiguracja:
\newcommand{\kurs}{Sterowniki robot\'{o}w}
\newcommand{\formakursu}{Projekt}

%odkomentuj właściwy typ projektu, a pozostałe zostaw zakomentowane
\newcommand{\doctype}{Za\l{}o\.{z}enia projektowe} %etap I
%\newcommand{\doctype}{Raport} %etap II
%\newcommand{\doctype}{Dokumentacja} %etap III

%wpisz nazwę projektu
\newcommand{\projectname}{Theremin}

%wpisz akronim projektu
\newcommand{\acronim}{-}

%wpisz Imię i nazwisko oraz numer albumu
\newcommand{\osobaA}{Cyprian \textsc{Hryniuk}, 235512}
%w przypadku projektu jednoosobowego usuń zawartość nowej komendy
\newcommand{\osobaB}{Tomasz \textsc{Mas\l{}o\'n}, 235827}

%wpisz termin w formie, jak poniżej dzień, parzystość, godzina
\newcommand{\termin}{srTN17}

%wpisz imię i nazwisko prowadzącego
\newcommand{\prowadzacy}{mgr in\.{z}. Wojciech \textsc{Domski}}

\documentclass[10pt, a4paper]{article}

\include{preambula}
\graphicspath{{./img/}}

\begin{document}

\def\tablename{Tabela}	%zmienienie nazwy tabel z Tablica na Tabela

\begin{titlepage}
	\begin{center}
		\textsc{\LARGE \formakursu}\\[1cm]		
		\textsc{\Large \kurs}\\[0.5cm]		
		\rule{\textwidth}{0.08cm}\\[0.4cm]
		{\huge \bfseries \doctype}\\[1cm]
		{\huge \bfseries \projectname}\\[0.5cm]
		{\huge \bfseries \acronim}\\[0.4cm]
		\rule{\textwidth}{0.08cm}\\[1cm]
		
		\begin{flushright} \large
		\emph{Skład grupy:}\\
		\osobaA\\
		\osobaB\\[0.4cm]
		
		\emph{Termin: }\termin\\[0.4cm]

		\emph{Prowadzący:} \\
		\prowadzacy \\
		
		\end{flushright}
		
		\vfill
		
		{\large \today}
	\end{center}	
\end{titlepage}

\newpage
\tableofcontents
\newpage

%Obecne we wszystkich dokumentach
\section{Opis projektu}
\label{sec:OpisProjektu}

Projekt ma na celu stworzenie theremina, w którym rolę anten spełniać będą dwa czujniki odległości: jeden do określania częstotliwości fali dźwiękowej, a drugi do jej amplitudy. 

\begin{figure}[H]
	\centering
	\includegraphics[width=0.5\textwidth]{architektura.png}
	\caption{Architektura systemu}
	\label{fig:Architektura}
\end{figure}

\section{Założenia projektowe}
%?????

%Obecne we wszystkich dokumentach
\section{Konfiguracja mikrokontrolera}

Tutaj powinna znaleźć się konfigurację poszczególnych peryferiów 
mikrokontrolera -- jeśli wykorzystywany jest np. ADC to należy 
podać jego konfigurację nie zapominając o DMA jeśli jest 
wykorzystywane. Proszę wzorować się na raporcie wygenerowanym 
z programu STM32CubeMx 
(plik PDF i TXT, Project -> Generate Report Ctrl+R). 
W pliku PDF jest to rozdział \textit{IPs} and \textit{Middleware Configuration}. 
Należy umieścić uproszczoną konfiguracje peryferiów w formie 
tabelek (najistotniejsze parametry + parametry zmienione, pogrubione).
Dodatkowo w pliku tekstowym (TXT) znajduje się konfiguracja pinów 
mikrokontrolera, którą również należy zamieścić w raporcie.

W przypadku, gdy projekt zakłada wykorzystanie większej liczby modułów
sekcję tą należy podzielić na odrębne podsekcje.

\begin{figure}[H]
	\centering
	\includegraphics[width=0.8\textwidth]{obraz.png}
	\caption{Konfiguracja wyjść mikrokontrolera w programie STM32CubeMX}
	\label{fig:KonfiguracjaMikrokontrolera}
\end{figure}

\newpage
\begin{figure}[H]
	\centering
	\includegraphics[width=0.9\textheight,angle=90]{obraz.png}
	\caption{Konfiguracja zegarów mikrokontrolera}
	\label{fig:KonfiguracjaZegara}
\end{figure}

%Obecne we wszystkich dokumentach
\subsection{Konfiguracja pinów}

\begin{table}[H]
	\centering
	\begin{tabular}{|l|l|l|l|}
		\hline
		Numer pinu	&	PIN & Tryb pracy & Funkcja/etykieta\\
		\hline
		2&	PC13 & ANTI\_TAMP	GPIO\_EXTI13	&B1 [Blue PushButton]\\
		3&	PC14 & OSC32\_IN*	RCC\_OSC32\_IN	&\\
		4&	PC15 & OSC32\_OUT*	RCC\_OSC32\_OUT	&\\
		5&	PH0&  OSC\_IN*	RCC\_OSC\_IN	&\\
		6&	PH1&  OSC\_OUT*&		RCC\_OSC\_OUT	\\
		16&	PA2&	USART2\_TX&	USART\_TX\\
		17&	PA3&	USART2\_RX&	USART\_RX\\
		21&	PA5&	GPIO\_Output&	LD2 [Green Led]\\
		29&	PB10&	I2C2\_SCL&	I2C\_SCL\\
		41&	PA8&	TIM1\_CH1&	PWM1\\
		46&	PA13*&	SYS\_JTMS-SWDIO&	TMS\\
		49&	PA14*&	SYS\_JTCK-SWCLK&	TCK\\
		55&	PB3*&	SYS\_JTDO-SWO&	SWO\\
		62&	PB9&	I2C2\_SDA&	I2C\_SCL\\
		\hline
	\end{tabular}
	\caption{Konfiguracja pinów mikrokontrolera}
	
\end{table}

%Obecne we wszystkich dokumentach
\subsection{USART}

Przykładowa konfiguracja peryferium interfejsu szeregowego.
Należy opisać do czego będzie wykorzystywany interfejs. 
Zmiany, które odbiegają od standardowych w programie CubeMX 
powinn być zaznaczone innym kolorem, jak to zostało pokazane 
w tabeli \ref{tab:USART}.

\begin{table}[H]
	\centering
	\begin{tabular}{|l|c|} \hline
		\textbf{Parametr} & Wartość \\
		\hline
		\hline  \textbf{Baud Rate}&11520  \\\hline
		\textbf{Word Length } & \textcolor{blue}{8 Bits (including parity)}\\\hline
		\textbf{Parity} &  None\\
		\hline
		\textbf{Stop Bits}& 1\\
		\hline
	\end{tabular}
	\caption{Konfiguracja peryferium USART}
	\label{tab:USART}
\end{table}


%Obecne w dokumencie do etapu I
\section{Harmonogram pracy}

Należy wstawić diagram Gantta oraz określić ścieżkę 
krytyczną. Ponadto zaznaczyć i opisać kamienie milowe.

\begin{figure}[H]
	\centering
	\includegraphics[width=0.5\textwidth]{obraz.png}
	\caption{Diagram Gantta}
	\label{fig:DiagramGantta}
\end{figure}

%Obecne w dokumencie do etapu I
\subsection{Podział pracy}

\textbf{Każdy z członków grupy powinien w każdym etapie mieć wymienione od 2 do 4 zadań.}
Przykładowa tabele podziału zadań dla etapu II 
(Tab. \ref{tab:PodzialPracyEtap2}) oraz dla etapu III 
(Tab. \ref{tab:PodzialPracyEtap3})
zostały przedstawione poniżej. 
Przy podziale prac nie uwzględniamy tworzenia dokumentacji projektu!

Przykładowy podział prac dla projektu pod tytułem 
"Automatyczny dyktafon rozmowy":

\begin{table}[H]
	\centering
	\begin{tabular}{|L{7cm}|L{0.8cm}||L{7cm}|L{0.8cm}|}
		\hline
		\hline
		\textbf{Tomasz Masłoń} & 
		\% & 
		\textbf{Cyprian Hryniuk} & \%\\
		\hline
		\hline
		Wstępna konfiguracja peryferiów w programie CubeMx		& &	
		 	Wstępna konfiguracja peryferiów w programie CubeMx	&\\
		\hline
		Implementacja obsługi Audio DAC & &
		 	Implementacja obsługi czujników odległości&\\
		\hline
		Opracowanie algorytmu modulującego falę dźwiękową na podstawie danych z czujników odległosci & &
		Opracowanie algorytmu modulującego falę dźwiękową na podstawie danych z czujników odległosci & \\
		\hline
		\end{tabular}
	\caption{Podział pracy -- Etap II}
	\label{tab:PodzialPracyEtap2}
\end{table}

\begin{table}[H]
	\centering
	\begin{tabular}{|L{7cm}|L{0.8cm}||L{7cm}|L{0.8cm}|}
		\hline
		\hline
		\textbf{Tomasz Masłoń} & 
		\% & 
		\textbf{Cyprian Hryniuk} & \%\\
		\hline
		\hline
		Finalna konfiguracja peryferiów w programie CubeMX		& &	
		Finalna konfiguracja peryferiów w programie CubeMX &\\
		\hline
		Opracowanie funkcji modyfikujących dźwięk  & &
		Opracowanie funkcji modyfikujących dźwięk &\\
		\hline
		Obsługa wyświetlacza ciekłokrystalicznego & &
		Obsługa joysticka & \\
		\hline
	\end{tabular}
	\caption{Podział pracy -- Etap III}
	\label{tab:PodzialPracyEtap3}
\end{table}


%Obecne we wszystkich dokumentach
\section{Podsumowanie}

Krótkie podsumowanie projektu

\newpage
\addcontentsline{toc}{section}{Bibilografia}
\bibliography{bibliografia}
\bibliographystyle{plabbrv}


\end{document}
